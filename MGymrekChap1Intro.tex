\iffalse \bibliography{MGymrekRefs.bib} \fi

\chapter{Introduction}

\section{Overview}

% Goal of genomics
A central goal in genomics is to understand the genetic variants that underlie phenotypic changes and lead to disease. Recent studies have identified thousands of genetic loci associated with human phenotypes. However, the majority of genome-wide studies have focused on single nucleotide polymorphisms (SNPs), which are both the simplest type of variation to genotype and one of the easiest to model. However, a wide range of  classes of variants is important for controlling human traits.

% STRs as a model
Short tandem repeats (STRs) consist of periodic DNA motifs of 1-6bp and comprise more than 1\% of the human genome. Their repetitive structure induces DNA polymerase slippage events that add or delete repeat units, resulting in mutation rates that are orders of magnitude higher than those for most other variant types. STRs are implicated in more than 40 human diseases, primarily in Mendelian disorders caused by large expansions of trinucleotide repeats. Additionally, several dozen single gene studies have shown that STRs may be involved in quantitative traits including gene expression. Because of their abundance, high polymorphism rates, and previous implication of functional roles, we focused on STRs as a model to investigate the role of complex variants in human phenotypes.

% What I present as our contribution
Here I present our contributions to enable the first large scale studies of STR variation and reveal that these loci play a significant role in complex traits in humans. In the first part of this thesis, we develop novel tools for high-throughput STR analysis from next generation sequencing data (\textbf{\autoref{chap:lobstr}, \autoref{chap:pbv}}). We then apply these methods to large sequencing cohorts consisting of thousands of samples with diverse origins to provide the first population-wide catalog of hundreds of thousands of previously uncharacterized STR loci (\textbf{\autoref{chap:catalog}, \autoref{chap:sgdp}}). An important aspect of this work has been a commitment to maximize the utility of our results for the wider genomics community through providing open-source software packages and online visualization tools that are already being utilized by other researchers. In the second part of this thesis, we interrogate the role of STRs in complex traits, focusing on gene expression as an initial phenotype (\textbf{\autoref{chap:estr}}). This study reveals more than 2,000 STRs whose lengths are correlated with gene expression (termed ``expression STRs'', or eSTRs) and shows that STRs make a significant contribution to regulating expression of nearby genes. These loci are enriched in putative regulatory regions and are predicted to modulate regulatory activity. These results highlight the contribution of STRs to the genetic architecture of gene expression and complex traits in humans.

% To frame this work
To frame this work, I first review properties and applications of STRs and what we know about their contribution to disease and molecular phenotypes in humans. Next, I describe challenges in developing high throughput methods for STR analysis and state of the art experimental and bioinformatic methods for doing so. I summarize what we have learned to date about patterns of STR variation in humans using these methods. Then, I review what has been revealed about the genetic architecture of gene expression through SNP studies. I examine evidence that suggests variants such as STRs that are not well tagged by common SNPs may play an important role in these traits. Finally, I summarize the contributions of this thesis toward enabling large scale STR analysis and highlighting an important role for STRs in human phenotypes.

\section{What is an STR?}%{STRs are an abundant source of genetic variability}
For the purposes of this thesis, STRs are defined as short motifs of 1-6bp repeated in tandem for a total length of $\sim$8bp or more. Note however that much of our work has excluded ``homopolymers'' with a motif length of 1 due to the difficulty in genotyping these loci. \autoref{chap:catalog} gives a more precise definition of the requirements for a sequence to be called an STR and Table \ref{tab:sgdptab1} shows the relative abundance of each motif length. Using our definition, the human genome harbors more than one million STRs, enriched near genes and promoters \cite{SawayaBagshawBuschiazzoEtAl2013}, comprising over 1\% of the genome. This is likely an underestimate, as large regions inaccessible to current sequencing technologies are highly enriched for long STRs (see \autoref{chap:conc}).

The repetitive nature of STRs induces DNA polymerase slippage events that add or delete repeat units, resulting in mutation rates of $10^{-3}-10^{-4}$ per locus per generation, orders of magnitude higher than those for most other variant types \cite{Ellegren2004,WeberWong1993}. As a result, STRs tend to be extremely polymorphic, providing a large source of genetic, and potentially phenotypic, variability in humans. Methods for capturing this variability across indviduals on a large scale, and for assessing the affect of this variation on phenotype, are the main focus of this work.

%Classical theoretical models have assumed that STRs mutate via a random walk, or ``simple stepwise model'' whereby a locus is equally likely to gain or lose one or more repeats at each mutation \cite{Slatkin1995}. A study by Sun \emph{et al} \cite{SunHelgasonMassonEtAl2012} analyzing hundreds of individual STR mutations in trios found clear evidence that longer STR alleles are more mutagenic and more likely to mutate to shorter alleles, whereas shorter alleles were likely to expand, showing allele length plays a key role in governing the mutation process. Additional studies have found links between sequence features including STR length, motif length, base composition, and presence of STR sequence imperfections on STR heterozygosity \cite{ODushlaineShields2008}, suggesting these features affect the mutation process.

\section{Applications of STRs}
STRs are an extremely abundant source of genetic polymorphism between individuals. Before the era of next generation sequencing, it was far cheaper to genotype STR lengths rather than simple sequence variations. As a result, STRs have been the marker of choice for a number of human genetics applications:
\begin{itemize}
\item \textbf{Linkage analysis}: Linkage analysis is a method to map the chromosomal location of genetic variants associated with disease and other phenotypes. It relies on the fact that loci that are nearer to each other are less likely to be separated by recombination. Given a set of markers spaced along a chromosome and observed crossover events, one can narrow down an interval in which the locus of interest lies. This method was used for decades, before the widespread use of DNA sequencing, and so relied on either STR markers or restriction fragment length polymorphisms. The Marshfield Panel \cite{BromanMurraySheffieldEtAl1998} contains thousands of STRs widely used in linkage analysis, and used throughout this work as a high quality validation set of STRs. 
\item \textbf{Forensics}: Unlike SNPs, which nearly always have only two alleles segregating in the population, STRs often have ten or or more common alleles consisting of different repeat numbers. Therefore, the information content in a single STR is quite high, and genotypes for a small number of STRs can often identify a single individual (except of course in the case of identical twins). During the 1980s, the FBI Laboratory developed the CODIS set, consisting of 13 STR loci plus the AMEL marker to determine sex \cite{BudowleSheaNiezgodaEtAl2001}. The National DNA Index database consists of more than 10 million CODIS profiles collected from arrested or convicted offenders. 
\item \textbf{Genetic genealogy}: Due to their high information content, STRs have also been widely used by genealogists to infer genetic relationships between individuals. In particular, Y-chromosome STRs have proven useful for studying patrilineal ancestry due to the co-inheritance of Y chromosomes with surnames in Western societies. For example, if a man with the last name ``Smith'' has a son, he generally passes on both his Y chromosome and his surname. Genealogists have taken advantage of this fact and companies such as Family Tree DNA and others have compiled large databases connecting Y-STR profiles with surnames. By searching these databases with a male's Y-STR profile, one can reveal surnames of patrilineally related individuals. Y-STR profiles are also widely used in paterntiy testing or to identify male subjects in forensics cases.
\end{itemize}

Because of their identifying information, genotyping STRs and storing large databases of STR profiles raises significant privacy concerns. For instance, searching genealogy databases with the Y-STR profile of an unidentified male may identify a unique surname. Combined with additional information such as age or state of residence, this can in some cases narrow down the male's identity to a single individual. Indeed, there are several documented cases of male children conceived by anonymous sperm donation finding their biological fathers by genotyping their own Y-STRs \cite{Motluk2005,Lehmann-Haupt2010}. This has important implications for human genetics studies, in which sample donors are assumed anonymous. We have shown that in many cases we could use Y-STRs obtained from publicly available whole genome sequencing datasets in addition to pedigree and geographical information to uniquely identify these individuals. This work is described in \autoref{chap:surname}. 

\section{Evidence of a regulatory role for STRs}
Multiple \emph{in vitro} studies have shown that STRs may regulate transcription. For instance, they have been shown to modulate transcription factor binding \cite{ContenteDittmerKochEtAl2002,MartinMakepeaceHillEtAl2005}. distances between promoter elements \cite{WillemsPaulHeideEtAl1990,YogevRosengartenWatson-McKownEtAl1991}, and splicing efficiency \cite{HefferonGromanYurkEtAl2004,HuiHungHeinerEtAl2005}. It has also been shown that certain STRs may induce DNA to form non-canonical ``Z''-DNA secondary structure, which can have an effect on transcriptional regulation of nearby genes \cite{RothenburgKoch-NolteRichEtAl2001}.

Additionally, \emph{in vivo} studies in model organisms have reported specific examples of STRs that modulate gene expression: Experimental modification of the number of repeats in the promoter of the \emph{FLO1} gene in yeast shows a direct quantitative effect on the cells' adherence to plastic \cite{VerstrepenJansenLewitterEtAl2005}; An intronic expanded GAA repeat in \emph{Arabidopsis thaliana} has an effect on growth \cite{SureshkumarTodescoSchneebergerEtAl2009}. A 5' UTR STR in \emph{avpr1a} predicts differences in socio-behavioral traits in prairie vole and modulates gene expression \emph{in vitro} \cite{HammockYoung2005}; Finally, STR length variations in or nearby coding regions of \emph{Alx-4} and \emph{Runx-2} were shown to correlate with facial and limb lengths in certain canine species \cite{FondonGarner2004}.

These single gene studies show that STRs could potentially provide a substrate for rapid evolution of fine-tunable gene expression regulation. Indeed, comparative genomics studies have found that the presence of STRs in promoters or transcribed regions is strongly associated with divergence of gene expression profiles across great apes \cite{SonayCarvalhoRobinsonEtAl2015}. This agrees with an earlier study showing similar trends across yeast strains \cite{VincesLegendreCaldaraEtAl2009}. 

Taken together, these anecdotes suggest that gene regulation by STRs is a widespread phenomenon that occurs across a range of taxa.

\section{STRs in human disease and phenotypic variation}
Dozens of STRs have been implicated in both disease and molecular phenotypes in humans. In nearly all of these cases, the resulting phenotype showed a quantitative relationship with the number of repeats, suggesting STRs may plan an important role in more complex quanitative traits.

\subsection{Dozens of disorders are caused by STR expansions}
% intro, examples
STR expansions are implicated in dozens of human single-gene, Mendelian disorders \cite{Mirkin2007}, affecting more than hundreds of thousands of patients in the U.S. \cite{CoffeeKeithAlbizuaEtAl2009}. The majority of these are due to expansions of trinucleotide repeats, and nearly all affect neurological function, most with a late onset disease course. For instance, an exonic CAG repeat expansion encoding polyglutamine results in Huntington's Disease, a devastating neurological disorder \cite{Mirkin2007}; a CGG expansion disrupts a methylation site on the X chrmosome leading to Fragile X Syndrome, one of the leading cuases of mental retardation in males \cite{LyonLaverYuEtAl2010}; a CUG expansion in the 3' untranslated (UTR) region of \emph{DPMK} results in myotonic dystrophy \cite{BrookMcCurrachHarleyEtAl1992}, a severe multisystemic form of muscular dystrophy. Other classes of repeats have also been implicated in STR expansion diseases. Recently, high throughput sequencing scans for causative SNPs fortuitously revealed that a hexanucleotide expansion in \emph{C9orf72} is responsible for 9p21-linked amyotrophic lateral sclerosis-frontotemporal dementia (ALS-FTD) \cite{RentonMajounieWaiteEtAl}. A summary of STR expansion diseases is given by Mirkin \cite{Mirkin2007}. %shown in \textbf{Table \ref{tab:intro1}}.

% TODO complete table
%\begin{table}[h!]
%\label{tab:intro1}
%\centering
%\begin{tabular}{c c c c c c}
%\hline
%Disorder & Gene & Repeat & Normal range & Pathogenic range & References \\
%\hline
%\hline
%\end{tabular}
%\caption{\textbf{An overview of STR expansion disorders.}}
%\end{table}

% mechanism?
The mechanisms by which most of these expansions lead to disease are still poorly understood. In cases of exonic repeat expansions, particularly polyglutamine expansions (e.g. in Huntington's Disease), it is thought that the expanded amino acid tracts form toxic aggregates that accumulate over time, consistent with the late-onset nature of these diseases \cite{MichalikVanBroeckhoven2003}. Alternatively, it was recently shown that a key factor in Huntington's Disease may be repeat-length dependent aberrant splicing of \emph{HTT} \cite{SathasivamNeuederGipsonEtAl2013}. Other proposed pathogenic mechanisms include loss of protein expression, over-expression of the wildtype protein copy, and toxic gain of function of RNAs encoding expanded repeats \cite{Pearson2011}. Recent studies have found evidence that an alternative mechanism, ``Repeat-associated non-ATG translation'', termed ``RAN-translation'', may be responsible for rendering CAG repeats toxic. Under this phenomenon, expanded CAG repeats in RNA may be translated in the absence of an ATG start codon, and may produce transcripts under all possible reading frames \cite{Pearson2011}. These ``RAN'' transcripts have already been identified in patients with a variety of repeat disorders, including spinocerebellar ataxia type 8 (SCA8), myotonic dystrophy type 1 (DM1), Fragile-X tremor ataxia syndrome (FXTAS), and ALS-FTD \cite{ClearyRanum2014}.

% anticipation, correlation with repeat number
One hallmark of repeat disorders is the phenomenon of ``anticipation'' in which the severity of the condition tends to increase and the age of onset tends to decrease with each generation. This generally corresponds to an in increase in repeat number for the expanded allele in each generation. Many of these repeats experience a bi-phasic mutation process: under a certain number of repeats the region is stable with relatively low mutation rate. However, once a threshold length has been passed, the repeat will have an extremely high mutation rate and tend toward massive expansions \cite{BourgeoisCoffeyRiveraEtAl2009}. Interestingly, the number of repeats is often directly related to phenotype. For instance, there is a negative linear relationship between CAG repeat number and age of onset of Huntington's Disease \cite{Rubinsztein2002}. This suggests that unlike point mutations, which can serve as an ``on/off'' switch for Mendelian and other disorders, repeats possess a more fine-tunable mechanism to affect phenotype by adjusting the number of repeats on a quantitative, rather than a binary, scale.

% polymorphism in population 
Many such pathogenic STR expansions have been studied in depth, and clearly indicate a biological function for at least a subset of repetitive elements. However, the majority of repeats remain uncharacterized and little is known about the extent of polymorphism and allele ranges at these and other STRs in healthy individuals. As mentioned above, STRs are prone to replication slippage events that cause them to mutate rapdily. Generally, intermediate length STRs of $\sim<$200bp tend to mutate in a stepwise fashion, compared to larger pathogenic STRs that form unusual DNA secondary structures leading to massive expansions. We hypothesize that whereas longer unstable repeats radically disrupt function of a given locus, leading to severe mono-genic disorders, STRs in the intermediate range length may be responsible for fine-tuning genomic regulation and for contributing to more subtle variation, which could make an important contribution to more complex traits in humans.

\subsection{STRs in complex human traits}
Little is known about the role of STRs in more complex, polygenic traits. This is largely because until recently there was limited ability to systematically profile these loci on a large scale (see \ref{sec:intromethods}) and because SNP-based studies have limited ability to capture STR associations (see \ref{sec:introarch}). However, a small number of STR associations with complex traits have been reported: repeat length in the first exon of the androgen receptor correlates with risk of hepatocellular carcinoma risk in women \cite{YuYangYangEtAl2002}; a TC repeat in \emph{HMGA2} is associated with uterine leiomyomata and decreased height \cite{HodgeTCuencoHuyckEtAl2009}; a CAG repeat in \emph{KCNN3} is associated with cognitive performance in schizophrenia \cite{GrubeGerchenAdamcioEtAl2011}. However, no study has systematically evaluated the role of STRs in complex traits, which we attempt to initially characterize here.

\subsection{Mechanisms for STR involvement in genome regulation}
STRs are found in at least 5\% of human protein-coding genes \cite{ODushlaineEdwardsParkEtAl2005} and are abundant in intragenic regions and UTRs \cite{LiKorolFahimaEtAl2004}. These rapidly evolving elements provide an evolutionary substrate to incrementally affect gene activity without introducing major sequences changes. STRs have been demonstrated or hypothesized to affect gene regulation in several ways, summarized below. % and in \textbf{Table \ref{tab:intro2}}.

\subsubsection{Transcribed STRs}
% exonic
As described above, many putative pathogenic STRs lie in coding regions and may lead to disease through mechanisms such as protein or RNA aggregation, RAN-translation, or aberrant splicing. Alternative mechanisms could allow exonic STRs to affect protin function. For instance, STR mutations in coding regions could function to silence genes by introducing frameshift mutations leading to premature stop codons \cite{GemayelVincesLegendreEtAl2010}, especially in the case of non tri- or hexa-nucleotide repeats. Additionally, expansions or contractions could alter spacing between protein domains, leading to a change in function.

% intronic
Intronic repeats may also affect gene regulation. Lengths of intronic STRs have been shown to affect gene expression \cite{GebhardtZankerBrandt1999}, in some cases through altering transcription factor binding sites. For example, a TATC repeat in \emph{TH} affects binding of the transcription factor ZNF191 \cite{AlbaneseBiguetKieferEtAl2001}. Several studies have also shown that intronic STRs may regulate splicing efficiency in a repeat-specific manner. For instance, the length of an intronic TG repeat in \emph{CFTR} is directly related with inclusion of the adjacent exon \cite{HefferonGromanYurkEtAl2004}, which is likely to be due to effects of the repeat on RNA secondary structure. An intronic CA repeat in \emph{eNOS} was shown to regulate splicing by affecting binding of the splicing factor HnRNP L \cite{HuiStanglLaneEtAl2003}, and it was later shown that intronic CA repeats may have a widespread effect on splicing \cite{HuiHungHeinerEtAl2005}.

% UTRs
STRs in UTRs may affect gene regulation. For example, a CTG/CAG repeat in the 3'UTR of \emph{DMPK} is implicated in the repeat expansion disorder myotonic dystrophy 1, and is thought to sequester splicing factors leading to aberrant splicing \cite{KoscianskaWitkosKozlowskaEtAl2015}. 3'UTR repeats may also harbor microRNA binding sites, affecting gene regulation.

\subsubsection{STR lengths influence promoter and enhancer activity}
In addition to transcribed regions, STRs can affect the function of genomic regulatory elements through several mechanisms. AC dinucleotides are over-represented in predicted \emph{cis}-regulatory elements \cite{RockmanWray2002}. Recently, it was shown that dinucleotide repeats are a hallmark of enhancer elements in \emph{Drosophila} and human cell lines \cite{Yanez-CunaArnoldStampfelEtAl2014}. This suggests they may provide an abundant source of transcriptional regulation in these elements. Heidari, \emph{et al.} \cite{HeidariNarimanSalehFamEsmaeilzadeh-GharehdaghiEtAl2012} demonstrated that a GA repeat in the promoter of \emph{SOX5} can affect nucleosome processing, affecting downstream gene expression. This finding is supported by a study in yeast that found variation in repeat length had a strong effect on nucleosome positioning and gene expression in 25 of 33 randomly chosen STR-containing promoters \cite{VincesLegendreCaldaraEtAl2009}. STRs may also bind transcription factors and create a number of binding sites dependent on the number of repeats. Guillon \emph{et al.} found that the oncogenic EWSR1-FLI1 fusion protein formed in Ewing Sarcoma preferentially binds GGAA repeats \cite{GuillonTirodeBoevaEtAl2009}. Interestingly, a recently study found that a reported genome-wide association study (GWAS) signal for Ewing Sarcoma actually points to a SNP for which the alternate allele joins two adjacent GGAA repeats into one long repeat tract, resulting in overexpression of the nearby gene \emph{EGR2} \cite{GrunewaldBernardGilardi-HebenstreitEtAl2015}. In another example, a pentanucleotide repeat in the promoter of \emph{PIG3} creates a varying number of \emph{p53} binding sites, affecting downstream expression \cite{ContenteDittmerKochEtAl2002}.

% TODO complete table
%\begin{table}[h!]
%\centering
%\label{tab:intro2}
%\begin{tabular}{c c c c c}
%\hline
%Gene & STR location & Tissue & Direction of effect & Reference \\
%\hline
%\hline
%\end{tabular}
%\caption{\textbf{STRs implicated in gene regulation. Full references given in the main text}}
%\end{table}

Together, these examples show clear evidence that STRs may regulate quantitative traits, such as gene expression, transcription factor binding, and splicing efficiency, in a repeat-dependent manner and therefore are prime candidates for contributing to complex traits in humans.

\section{Methods for genotyping STRs}
\label{sec:intromethods}
\subsection{Capillary electrophoresis}
The current gold standard technique for STR typing relies on cumbersome capillary electrophoresis methods \cite{ButlerBuelCrivellenteEtAl2004}, which require PCR amplification of the locus of interest, followed by size separation via electrophoresis to determine the alleles present. The method is widely used for typing STRs used as genetic markers for linkage analyses (Marshfield set \cite{BromanMurraySheffieldEtAl1998}), the FBI CODIS set, markers for genealogical studies \cite{ZerjalXueBertorelleEtAl2003,SkoreckiSeligBlazerEtAl1997} and for determining alleles present at known pathogenic loci \cite{LyonLaverYuEtAl2010,BlancoSuarezGandia-PlaEtAl2008}.

Capillary electrophoresis uses a separate reaction per locus, and requires time-consuming optimization of conditions for each reaction. With a cost that is upwards of several dollars per STR and lengthy preparation, current panels can consist of up to several hundreds of STRs - only a fraction of the hundreds of thousands of STR loci in the human genome \cite{Benson1999}. While the method is robust and can achieve accuracy above 99\% \cite{WeberBroman2001}, it has several technical limitations. First, many loci, especially dinucleotide STRs, are plagued by ``stutter peaks'' due to errors introduced during PCR amplification that may complicate calling. Second, this technique can only return the size of the allele: it is unable to distinguish homoplasmic alleles \cite{WeberBroman2001} (two different alleles with the same size but distinct sequences). Third, because capillary techniques simply return the length of the amplified region, it will be sensitive to the presence of linked insertions or deletions that are not part of the STR itself being genotyped. Due to these limitations, electrophoresis is unable to provide an accurate genome-wide picture of sequence variation at STR loci.

\subsection{Genotyping STRs from next-generation sequencing}
\subsubsection{Challenges of genotyping STRs from short reads}
Although theoretically any type of variation should be captured by DNA sequencing, STRs have proven challenging to genotype from short reads produced by high-throughput sequencing platforms. Major challenges include:
\begin{enumerate}
\item Reads must entirely span an STR region to be informative about the number of repeats present.
\item STRs with a large length difference from the reference sequence present as a gapped alignment problem. The run time of mainstream aligners such as BWA \cite{LiDurbin2009a} increases rapidly  with the number of insertions or deletions allows.
\item As in capillary techniques, sequencing technologies require PCR amplification of the DNA sample, which can introduce false ``stutter peaks'' due to the same polymerase slippage process leading to germline STR mutations.
\end{enumerate}
As a result of these challenges, STRs are not routinely analyzed in sequencing studies \cite{TreangenSalzberg2012}, and loci containing repetitive loci are frequently filtered out due to the high presence of genotyping errors in these regions.

A major contribution of our work was develop the first efficient algorithm for generating accurate STR genotypes, called lobSTR \cite{GymrekGolanRossetEtAl2012} and described in \autoref{chap:lobstr}. Over the last several years, additional bioinformatic tools and long read sequencing technologies have arisen that have the potential to greatly increase STR calling. These are discussed in the conclusion (\autoref{chap:conc})

\subsubsection{Challenges in visualizing complex variatns}
A key aspect of developing and using tools for variant analysis is visualization of sequence alignments. Often, inspecting raw read alignments can be informative of systamic sequencing artifacts leading to errneous genotype calls.Currently, the UCSC Genome Browser \cite{KentSugnetFureyEtAl2002} and the Integrative Genomics Viewer (IGV) \cite{RobinsonThorvaldsdottirWincklerEtAl2011} are the most widely used genome browsers for alignment visualization.

Visualization of insertions and deletions are key to analyzing reads containing STR variations from the reference genome. However, UCSC, IGV, and similar tools are limited in their ability to display insertions from the reference genome. Because the display is based entirely on the reference genome, insertions are simply displayed as a vertical bar, with no information about the length of insertions. As a result, reads containing different insertions consisting of different lengths, such as a diploid locus where both alleles are longer than the referene allele, will be displayed identically. To overcome this and other genome browser challenges, I have created a novel application, PyBamView, for visualizing sequence alignments at complex variants. This contribution is described in \autoref{chap:pbv}.

\section{Population-wide characterization of STR variation}
Several large panels of STR variation have been previously generated using capillary electrophoresis. To the best of our knowledge, the largest panels are from the Rosenberg Lab (\url{https://rosenberglab.stanford.edu/data/rosenbergEtAl2005/}) which contains genotypes for 993 STRs in 1,048 individuals from the Human Genome Diversity Project (HGDP) (\url{http://www.hagsc.org/hgdp/}) and the Payseur Lab (\url{http://payseur.genetics.wisc.edu/strpData.htm}), with 721 STRs in 201 individuals from the HapMap Project. Both panels use a subset of the Marshfield marker set originally used for linkage analysis. In addition to autosomal STRs, 61 Y-STRs have been genotyped in 669 HGDP samples (\url{ftp://ftp.cephb.fr/hgdp_supp9/}) and for 16 Y-STRs in 49 HapMap Samples \cite{HeGitschierZerjalEtAl2009}.

These panels have shown that STRs are informative of ancestry and can accurately capture population structure. Rosenberg \emph{et al.} found that STRs could be used to cluster individuals by geographic regions and individual populations \cite{RosenbergPritchardWeberEtAl2002}, in strong agreement with self-reported ancestries. He \emph{et al.} showed that Y-STRs were informative of geographic region of origin of HapMap samples \cite{HeGitschierZerjalEtAl2009}. Additionally, Y-STRs can be informative of specific historical events. For instance, Y-STR analysis has identified haplotypes likely to have descended from Genghis Khan \cite{ZerjalXueBertorelleEtAl2003} and haplotypes specific to the Cohen group of Jewish priests \cite{SkoreckiSeligBlazerEtAl1997}.

Capillary electrophoresis panels have been valuable in inferring a number of features of STR variability, such as the dependence of mutation rate on repeat unit length \cite{JarveZhivotovskyRootsiEtAl2009} and repeat tract length \cite{SunHelgasonMassonEtAl2012}. However, the STRs used in these panels represent a small fraction of all STRs in the genome, and consist almost entirely of di- and tetranucleotide repeats. Moreover, they have been particularly chosen due to their high polymorphism rates and because they are straightforward to genotype, likely imposing ascertainment biases that may confound analyses \cite{ErikssonManicaScherer2011}.

High throughput sequencing holds the promise of creating unbiased genome wide catalogs of a variety of types of genetic variation. This technology has so far been used mostly to catalog variation at SNPs. For instance, the 1000 Genomes Project sequenced more than 2,500 individuals from diverse origins to capture the vast majority of common SNP variation across humans \cite{AbecasisAltshulerAutonEtAl2010}. However, next generation sequencing can theoretically capture any variant type, including STRs. Beyond the several thousand Marshfield, Y-STR, medical, and forensics loci that have been widely genotyped, the remaining more than 1 million human STRS remain almost completely uncharacterized. A major contribution of our work has been to provide the first look at population-wide polymorphism data for hundreds of thousands of STRs. This work is described in \autoref{chap:catalog} and \autoref{chap:sgdp}, and provides unprecedented power for novel STR analyses going forward.

\section{Genetic architecture of gene expression and complex traits}
\label{sec:introarch}

A fundamental goal of human genomics is to understand how genetic variation leads to changes in phenotypes. Genome wide association studies (GWAS) have uncovered thousands of loci associated with human traits. The vast majority of these are in non-coding regions, suggesting regulation of gene expression as a major underlying molecular mechanism. Large scale efforts have mapped thousands of common variants predicted to regulate gene expression in \emph{cis} and shed light on the heritability of gene expression.

\subsection{Expression quantitative trait loci}
Expression quantitative trait loci, or eQTLs, refer to variants that show a quantitative relationship with gene expression. In most cases, eQTLs refer to a bi-allelic variant, encoded as 0 (homozygous reference), 1 (heterozygous), or 2 (homozygous non-reference) that shows a linear relationship with expression. Importantly, just because a variant is a significant eQTL does not imply causality. In many cases an eQTL simply tags a nearby causal variant in linkage disequilibrium.

Increasingly large efforts have been conducted to map eQTLs in humans across a range of cell types. Most large studies so far have focused on lymphoblastoid cell lines (LCLs) (e.g. \cite{GaffneyVeyrierasDegnerEtAl2012} and the gEUVADIS Project \cite{LappalainenSammethFriedlanderEtAl2013}). Recently, the GTEx Project analyzed eQTLs across a broad range of tissues in hundreds of individuals \cite{ArdlieDelucaSegreEtAl2015}, providing a valuable community resource. Each of these studies has revealed thousands of eQTLs, with many more to be discovered. For instance, the GTEx Project (Figure 2A in \cite{ArdlieDelucaSegreEtAl2015}) found that the number of significant eQTLs increases linearly with sample size within the range of samples analyzed, implying the presence of thousands more eQTLs that they were underpowered to detect. Together, these results suggest that nearly every gene in the genome is influenced by one or more eQTL.

\subsection{Heritability of gene expression}
Although eQTLs have been quite successful in uncovering regulatory variants, individual eQTLs explain very little ($< 10\%$) overall variation in gene expression \cite{GrundbergSmallHedmanEtAl2012}. Heritability analyses can be informative of the genetic basis of gene expression.

Average heritability of gene expression varies across tissues but ranges from around 10-30\% as measured using identity by descent (IBD) analyses. For instance, average heritabilities of $\sim$0.2 for LCLs \cite{GrundbergSmallHedmanEtAl2012,DixonLiangMoffattEtAl2007}, $\sim$0.1 for blood \cite{PriceHelgasonThorleifssonEtAl2011,WrightSullivanBrooksEtAl2014}, $\sim$0.25 for adipose \cite{PriceHelgasonThorleifssonEtAl2011,GrundbergSmallHedmanEtAl2012}, and 0.16 for skin \cite{GrundbergSmallHedmanEtAl2012}. Notably, ``purer'' tissues such as adipose tend to have higher heritability than tissues that contain a mixture of cell types such as whole blood.

In most tissues studied, variance component analyses have estimated that \emph{cis} variation accounts for around 30\% of all heritability for each gene \cite{GrundbergSmallHedmanEtAl2012,PriceHelgasonThorleifssonEtAl2011}. In contrast, common SNP eQTLs explain only around 75\% of the total \emph{cis} heritability, leaving 25\% unexplained.  Orthogonally, it has was recently reported that haplotypes of common SNPs can explain substantially greater heritability of complex traits than common SNP genotypes alone. These results suggest that other \emph{cis} variants not tagged by individual common SNPs, such as STRs, rare variants, retrotransposons, and others, could play an important role and may explain some of the ``missing heritability'' of complex traits.

\section{Contributions of this thesis}
Taken together, the studies described above provide strong evidence that STRs may play a widespread role in quantitative traits in humans. At the outset of this work, STRs were not amenable to large scale studies due to limitations in bioinformatic and sequencing technologies. Furthermore, repetitive regions were largely considered neutral, or ``junk DNA'' that were simply filtered out of most analyses.

In this thesis, I present our work to enable and perform genome-wide analysis of the effect of STRs on gene expression, and ultimately complex traits, in humans. These efforts have shown that STRs indeed contribute significantly to the heritability of gene expression and likely have a widespread effect on genome regulation. Importantly, the tools and results generated by this work have helped to renew interest in variants beyond point mutations and have highlighted the importance of including repetitive variations in genome-wide studies of complex traits. 

Specifically, I present the following studies, briefly summarized here:

\subsection{Tools for STR analysis from short reads}
STRs pose major challenges to current bioinformatics pipelines. They are often extremely polymorphic and exhibit large length differences from the reference sequence. Additionally, they are prone to stutter errors that occur due to polymerase slippage during PCR amplification. As the first step to studying STRs on a large scale, we developed lobSTR \cite{GymrekGolanRossetEtAl2012} \autoref{chap:lobstr}, an algorithm for profiling short tandem repeats from high throughput sequencing data. lobSTR employs a unique mapping strategy to rapidly align repetitive STR containing reads, and uses statistical learning techniques to account for stutter noise.

We have invested significant effort in making this tool user friendly, and lobSTR now has an active online user community on Github and Google groups. lobSTR has been used to generate large scale catalogs of STR variation from the 1000 Genomes and other projects (see below). We continue to improve lobSTR to adapt to improvements in sequencing technology and bioinformatics tools. For instance, lobSTR now handles input from the BWA-MEM aligner, has a specific noise model for PCR-free protocols, and uses joint calling to simultaneously genotype thousands of samples at once.

In addition to lobSTR itself, we are actively developing additional tools for analysis and visualization of STR and other complex variant data. To deal with challenges in visualization STRs and other complex variants, I developed PyBamView \cite{Gymrek2014} \autoref{chap:pbv}, an interactive alignment visualization tool. PyBamView allows accurate visualization of insertions, which are not displayed well by current browsers such as IGV and UCSC, and allows researchers to share alignment views with others through a web browser.

\subsection{The first genome-wide catalogs of STR variation}
Until recently, studies of STR variation have been limited to several highly ascertained sets of loci encompassing several thousand highly polymorphic autosomal and Y-chromosome STRs used in linkage studies, genetic genealogy, and forensics. Little is known about patterns of polymorphism across the majority of the 1.6 million STR loci in the human genome. Building a catalog of STR variation is an important first step in determining their contribution to genetic and phenotypic variation.

To build an initial genome-wide STR catalog, we applied lobSTR to more than 3,000 low coverage whole genome sequencing datasets generated by the 1000 Genomes Project \cite{WillemsGymrekHighnamEtAl2014} \autoref{chap:catalog}. We used this call set to analyze sequence determinants of STR variation, assess patterns of variation in coding regions, and find common loss of function alleles. We found that hundreds of thousands of STRs are polymorphic in the number of repeats across individuals. This catalog has already served as a valuable resource for researchers interested in variation at specific STR loci that may be implicated in human phenotype. 

While our initial catalog gave accurate information on the distribution of STR alleles across populations, the quality of individual genotypes was poor due to very low sequencing coverage. Additionally, we were not able to obtain accurate genotypes for homopolymers, which show notoriously high error rates due to polymerase slippage. We recently created the most comprehensive STR call set to date based on 300 deeply sequenced genomes from the Simons Genome Diversity Project \autoref{chap:sgdp}. These calls show 93\% concordance with gold standard STR genotypes from capillary electrophoresis, and accurately capture population structure. This dataset provides unprecedented opportunities to study STR variation that were not possible using previous studies either due to the small number of markers or to the low quality of individual genotypes, including in-depth study of homopolymers and population dynamics of STR variation.

\subsection{Abundant contribution of STRs to gene expression in humans}
With a robust pipeline for STR genotyping and a large catalog of STR genotypes, we were in a position to assess the genome-wide contribution of STRs to phenotypes for the first time. We focused on the role of STRs in regulating gene expression. Multiple single-gene studies in humans and other organisms have shown that STRs may modulate gene expression in \emph{cis}. However there has been no systematic study of their effect on expression in humans.

We conducted a genome-wide analysis of STRs that affect expression of nearby genes, which we termed expression STRs (eSTRs), in lymphoblastoid cell lines (LCLs) \cite{GymrekWillemsGuilmatreEtAl2015} \autoref{chap:estr}. This well-studied model permitted the integration of whole genome sequencing data, expression profiles from RNA-sequencing and arrays, and functional genomics data. We tested for association in close to 190,000 STR$\times$gene pairs and found over 2,000 significant eSTRs. Using a multitude of statistical genetic and functional genomics analyses, we show that hundreds of these eSTRs are predicted to be functional, and that 10-15\% of \emph{cis} heritability of gene expression in LCLs can be attributed to STRs, uncovering a new class of regulatory variants.

\subsection{Identifying personal genomes by surname inference}
Sequencing datasets are often shared without identifiers under the assumption that the identities of the samples will not be revealed. We showed that we can recover genealogical STRs from the Y chromosome by applying lobSTR to whole genome sequencing datasets. We used the resulting genotypes to query recreational genealogical databases to recover the surnames of anonymous sample donors. Combining the surname with additional metadata such as age and state can lead to a complete breach of anonymity \cite{GymrekMcGuireGolanEtAl2013} \autoref{chap:surname}.

Due to the widespread use of STRs in forensics and genealogy, this work had important implications for genetic privacy of human research subjects. As a result I have become more aware of ethical concerns and have been an active participant in discussions of social aspects of genetic research in the wider genomics community. Importantly, it is now recognized that it is impossible to promise anonymity, and that this should be clarified to participants. As a result of our study many data usage agreements have been augmented to prevent researchers from attempting to identify participants. We believe this project paved the way toward greater transparency about data privacy issues and will ultimately help protect and inform participants in genetic research.

